\hypertarget{active}{%
\subsubsection{Active}\label{active}}

\textbf{NIH/NINDS R01 NS030678}\\
\emph{Comparison of Hemorrhagic \& Ischemic Stroke Among Blacks and
Whites}\\
Kleindorfer, PI (04/01/15 - 03/31/20)\\
Tracking of population-based stroke incidence in the Greater Cincinnati
and Northern Kentucky region, with special emphasis on stroke in the
young and stroke recurrence.\\
Role: Co-I

\textbf{Internal Processes and Methods Award - Center for Clinical \&
Translational Science \& Training}\\
\emph{Using Machine Learning to Supplement Electronic Health Record
databases with Individual Socioeconomic Status}\\
Brokamp, PI (9/1/17 - 6/30/20)\\
Retrospective epidemiological studies are often created using electronic
health record databases. Although these records are ``wide'', they are
not ``deep'' with respect to individual level demographic data. We
propose a novel machine learning based approach that uses open city and
auditor databases to predict individual level income and family
socioeconomic status. This will solve the urgent problem of
unconfounding for individual SES in the execution of EHR based
research.\\
Role: PI

\textbf{NIH 5UG3OD023282-02}\\
\emph{Children's Respiratory Research and Environment Workgroup
(CREW)}\\
Gern, PI (9/01/2016 - 8/31/2023)\\
This consortium will identify asthma endotypes and overcome shortcomings
of individual cohorts by providing a large (nearly 9000 births and
long-term follow-up of 6000-7000 children and young adults) and diverse
national data set, harmonizing data related to asthma clinical
indicators and early life environmental exposures, developing
standardized measures for prospective data collection across CREW
cohorts and other ECHO studies, and conducting targeted enrollment of
additional subjects into existing cohorts.\\
Role: Co-I

\textbf{NIH/NIEHS R21ES030092}\\
\emph{Developing and Evaluating Novel Strategies for Reporting Back
Individual Results of Personal Air Monitors}\\
Ryan, PI (9/1/19 - 9/1/21)\\
This project will work to develop new methods for reporting
individual-level personal air pollution concentrations to study subjects
to better help them understand the risk of air pollution and to modify
their behavior to improve health outcomes.\\
Role: Co-I

\textbf{NIH/NHLBI R01HL141286-01A1}\\
\emph{Mapping Environmental Contributions to Rapid Lung Disease
Progression in Cystic Fibrosis}\\
Sczcesniak, PI (1/1/19 - 12/31/23)\\
The overall objective of this research is to leverage a rich CF
registry, extant national and local environmental data sources, and
prospectively collected study data to accurately forecast the onset of
rapid decline progression.\\
Role: Co-I

\textbf{NIH/NIA R21AG057983}\\
\emph{A Novel Research Infrastructure Enabling Life-Course Studies of
Healthy Aging}\\
Woo/Urbina, PI (8/15/18 - 7/31/23)\\
The goal of this two-phase study is to develop the data and biospecimen
infrastructure for the Bogalusa Heart Study, the Princeton Lipid
Research Study and the NHLBI Growth and Health Study (R21 phase) and to
conduct pilot evaluations of the feasibility, acceptability and validity
of data collected using a variety of biometric sensors relating to
cardiometabolic risk, sleep quality and cognition in these cohorts (R33
phase). These two phases will together prepare these cohorts for future
aging-related studies.\\
Role: Co-I

\textbf{ECHO Opportunities and Infrastructure Fund Award}\\
\emph{Decentralized and Reproducible Geomarker Assessment for Multi-Site
Studies}\\
Brokamp, PI (09/01/2019 - 08/31/2021)\\
This award will work towards building geospatial exposure assessment
computing tools for utilizing high resolution spatiotemporal gridded
datasets within ECHO.\\
Role: PI

\textbf{Ohio Department of Health Contract No.~CSP907820}\\
\emph{Model Identifying Geographic Areas in Ohio for Blood Lead
Testing}\\
Brokamp, PI (4/15/2020 - 9/30/2020)\\
This award will develop a predictive model to determine which children
should be tested for potentially high blood lead during physician visits
based on their residential location.\\
Role: PI

\textbf{AHRQ PEDSnet K12}\\
\emph{Inpatient Screening for Parental Adversity and Resilience}\\
Shaw, PI (1/1/19 - 12/31/20)\\
This award will work to establish and implement a screening approach for
the assessment of parental adverse childhood experiences in the hospital
setting.\\
Role: Co-I

\hypertarget{pending}{%
\subsubsection{Pending}\label{pending}}

\textbf{NIH NLM 1R01LM013222-01A1}\\
\emph{A Framework for Automated and Reproducible Geomarker Curation and
Computation at Scale}\\
Brokamp, PI (4/1/20 - 3/31/24)\\
This award will create a framework for developing a standardized, free
and open source library of reproducible and computable geomarkers that
will enhance the efficiency and collaboration of biomedical researchers
utilizing place-based data at scale.\\
Role: PI

\textbf{NIH NLM 1R01LM013420-01}\\
\emph{Development of an Electronic Health Records to Enhanced Research
Database Pipeline With Applications in Intrauterine Substance
Exposures}\\
Brokamp/Hall, PI (7/1/20 - 6/30/24)\\
Our objective in this work is to develop an approach to allow biomedical
researchers to more fully harness EHR content, including structured and
free-text components, and to facilitate enhanced research dataset
creation through repeated, automated EHR extractions including automatic
integration of geomarkers and external datasets.\\
Role: Co-PI

\hypertarget{complete}{%
\subsubsection{Complete}\label{complete}}

\textbf{Internal Arnold W. Strauss Fellowship Award - Cincinnati
Children's Hospital}\\
\emph{Assessing Exposure to Air Pollution Across Time and Space}\\
Brokamp, PI (7/1/16 - 6/30/17)\\
The primary objective of this award is to combine satellite-based
measurements, land use characteristics, and meteorologic data to create
a hybrid spatiotemporal model for ground level exposure to particulate
matter using exact addresses and dates.\\
Role: PI

\textbf{Internal Processes and Methods Award - Center for Clinical \&
Translational Science \& Training}\\
\emph{Validating a Geocoding Approach for Multi Site Studies}\\
Brokamp, PI (1/24/17 - 6/30/17)\\
The primary objective of this award is to compare the geocoding
(assigning latitude and longitude coordinates to addresses) accuracy of
our software DeGAUSS (DEcentralized Geomarker Assessment for mUlti Site
Studies) to with other common geocoding software. Furthermore, each
method will be evaluated based on it ability to correctly estimate
environmental exposures and community-level characteristics.\\
Role: PI
